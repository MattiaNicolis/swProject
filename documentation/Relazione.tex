\documentclass[12pt, a4paper, oneside]{book}

\usepackage[utf8]{inputenc}
\usepackage[T1]{fontenc}
\usepackage[italian]{babel}
\renewcommand{\familydefault}{\sfdefault}

\usepackage[a4paper, margin=2.5cm]{geometry}

\usepackage{xcolor}
\definecolor{azzurrino}{HTML}{1565C0}
\definecolor{grigiochiaro}{HTML}{F5F5F5}
\definecolor{grigioscuro}{HTML}{424242}

\usepackage[
    colorlinks=true,
    linkcolor=grigioscuro,
    citecolor=azzurrino,
    urlcolor=azzurrino
]{hyperref}

\usepackage{titlesec}

\titleformat{\chapter}[display]
  {\normalfont\Huge\bfseries\color{azzurrino}}
  {\filleft\Huge\thechapter}
  {0pt}
  {\titlerule[1pt]\vspace{1ex}\filright}
  [\vspace{1ex}\titlerule]

\titleformat{\section}
  {\normalfont\Large\bfseries\color{azzurrino}}{\thesection}{1em}{}

\titleformat{\subsection}
  {\normalfont\large\bfseries\color{grigioscuro}}{\thesubsection}{1em}{}

\setcounter{tocdepth}{4}
\setcounter{secnumdepth}{4}

\titleformat{\subsubsection}
  {\normalfont\large\bfseries\color{grigioscuro}}{\thesubsubsection}{1em}{}

\usepackage[font=small, labelfont={bf, color=azzurrino}]{caption}

\usepackage{fancyhdr}
\setlength{\headheight}{25pt}
\pagestyle{fancy}
\fancyhf{}
\fancyhead[L]{\color{grigioscuro}\leftmark}
\fancyhead[R]{\color{azzurrino}\thepage}
\renewcommand{\headrulewidth}{0.4pt}
\renewcommand{\headrule}{\hbox to\headwidth{\color{azzurrino}\leaders\hrule height \headrulewidth\hfill}}

\usepackage{enumitem}
\setlist{nosep, leftmargin=1.5em}

\usepackage{graphicx}
\usepackage{bookmark}
\usepackage{float}
\usepackage{wrapfig}
\usepackage[normalem]{ulem}
\usepackage{subcaption}
\usepackage{soul}
\usepackage{parskip}
\usepackage{mdframed}

\usepackage[most]{tcolorbox}
\tcbuselibrary{listings, skins, breakable, theorems}
\usepackage{amsmath}
\usepackage{amssymb}

\usepackage{longtable}
\usepackage{array}
\usepackage{tabularx}

\title{\Huge\bfseries Progetto di ingegneria del software \\[0.5ex]
\textcolor{azzurrino}{Glicocare} \\[1ex]
\Large UniVR - Dipartimento di Informatica \\[1.5ex]
\Large A.A. 2024/25}
\author{
\Large{Marco Broccolato -- VR501013}\\[0.6em]
\Large{Mattia Nicolis -- VR500356}\\[0.6em]
\Large{Simone Pedretti -- VR500040}
}
\date{}

% =================================
\begin{document}
    \frontmatter
    \maketitle
    \tableofcontents
    \thispagestyle{empty}
    \newpage

    %===========================
    \mainmatter

    % --- CAPITOLO 1 - RISCRITTO SPECIFICHE DEL PROGETTO ---
    \chapter{Specifiche e analisi dei requisiti}
    Il sistema proposto consente l’accesso a personale medico e pazienti affetti da diabete.\\[0.2ex]
    Gli utenti sono classificati in due ruoli distinti: \textcolor{azzurrino}{\textbf{Diabetologo}} e \textcolor{azzurrino}{\textbf{Paziente}}.\\[0.2ex]
    L’accesso al sistema avviene tramite autenticazione mediante credenziali predefinite, fornite dagli amministratori di sistema, che determinano il caricamento della schermata iniziale associata al ruolo dell’utente autenticato.

    \begin{figure}[H]
        \centering
        \includegraphics[width=1.1\linewidth]{../documentation/images/casi-uso-Totale.png}
        \caption{Casi d'uso - diabetologo e paziente}
    \end{figure}
    \section{Use case}

    \subsection{Diabetologo}
    A seguito di una corretta autenticazione, il diabetologo accede alla propria home page, dalla quale può visualizzare i propri dati anagrafici e la lista dei pazienti affetti da diabete.

    \medskip
    \begin{tcolorbox}[
        title=$\textbf{UC}_\text{d}$ -- Visualizza dati paziente,
        colback=white,
        colframe=black,
        fonttitle=\bfseries
    ]

        \textbf{Attori:} diabetologo
        
        \medskip
        \textbf{Precondizioni:} diabetologo ha effettuato correttamente l’autenticazione al sistema
        
        \medskip
        \textbf{Flusso principale:}
        \begin{enumerate}[label=$\triangleright$]
            \item diabetologo accede al sistema
            \item sistema mostra l’area riservata del diabetologo
            \item diabetologo seleziona un paziente
            \item sistema consente al diabetologo di:
            \begin{itemize}[label=$\diamond$]
                \item creare, modificare o eliminare una terapia
                \item visualizzare l’andamento della glicemia e del peso corporeo
                \item aggiungere o rimuovere dati dallo storico clinico del paziente
                \item visualizzare lo storico dei questionari compilati
            \end{itemize}
            \item diabetologo visualizza i questionari non conformi
            \item diabetologo visualizza i pazienti con valori glicemici fuori dal range
        \end{enumerate}
        
        \medskip
        \textbf{Postcondizioni:} dati visualizzati o modificati risultano aggiornati nel sistema
    \end{tcolorbox}

    \begin{figure}[H]
        \centering
        \includegraphics[width=0.5\linewidth]{../documentation/images/activity-diagram-modifcaDatiPaziente.jpg}
        \caption{Diagramma di sequenza - Ricevi Notifica}
    \end{figure}

    \begin{tcolorbox}[
        title = $\text{UC}_\text{d}$ - Ricevi notifica,
        colback=white,
        colframe=black,
        fonttitle=\bfseries
    ]
        \textbf{Attori:} diabetologo

        \medskip
        \textbf{Precondizioni:} diabetologo ha effettuto correttamente l'auteniticazione al sistema

        \medskip
        \textbf{Flusso principale:}
        \begin{itemize}[label=$\triangleright$]
            \item diabetologo accede al sistema
            \item sistema mostra l'area riservata del siabetologo
            \item diabetologo clicca sul bottone "... Mail"
            \item sistema mostra l'area riservata alla mail
            \item sistema consente al diabetologo di:
            \begin{itemize}[label=$\diamond$]
                \item inviare o rispondere a una mail
                \item inviare una mail al paziente che ha compilato in maniera errata il questionario
            \end{itemize}
        \end{itemize}

        \medskip
        \textbf{Postcondizioni:}
    \end{tcolorbox}

    \begin{figure}[H]
        \centering
        \includegraphics[width=0.7\linewidth]{../documentation/images/activity-diagram-riceviNotifica.jpeg}
        \caption{Diagramma di sequenza - Modifica dati paziente}
    \end{figure}

    \subsection{Paziente}
    A seguito di una corretta autenticazione, il paziente accede alla propria home page, dalla quale può visualizzare i propri dati anagrafici, i grafici relativi all’andamento della glicemia e del peso corporeo, le terapie assegnate e lo storico dei questionari compilati.

    \begin{tcolorbox}[
        title = $\text{UC}_\text{p}$ - Misurazione glicemia,
        colback=white,
        colframe=black,
        fonttitle=\bfseries
    ]
        \textbf{Attori:} paziente

        \medskip
        \textbf{Precondizioni:} paziente ha effettuto correttamente l'auteniticazione al sistema

        \medskip
        \textbf{Flusso principale:}
        \begin{itemize}[label=$\triangleright$]
            \item paziente accede al sistema
            \item sistema mostra l'area riservata del paziente
            \item paziente inserisce il valore della glicemia, il momento della misurazione e l'ora
        \end{itemize}

        \medskip
        \textbf{Postcondizioni:}
    \end{tcolorbox}

    \medskip

    \begin{tcolorbox}[
        title = $\text{UC}_\text{p}$ - Misurazione del peso corporeo,
        colback=white,
        colframe=black,
        fonttitle=\bfseries
    ]
        \textbf{Attori:} paziente
        
        \textbf{Precondizioni:} paziente ha effettuto correttamente l'auteniticazione al sistema
        
        \textbf{Flusso principale:}
        \begin{itemize}[label=$\triangleright$]
            \item paziente accede al sistema
            \item sistema mostra l’area riservata del paziente
            \item paziente inserisce il valore del peso corporeo
        \end{itemize}

        \medskip
        \textbf{Postcondizioni:}
    \end{tcolorbox}

    \medskip
    \begin{tcolorbox}[
        title = $\text{UC}_\text{p}$ - Compilazone questionario,
        colback=white,
        colframe=black,
        fonttitle=\bfseries
    ]
        \textbf{Attori:} paziente
        
        \textbf{Precondizioni:} paziente ha effettuto correttamente l'auteniticazione al sistema
        
        \textbf{Flusso principale:}
        \begin{itemize}[label=$\triangleright$]
            \item paziente accede al sistema
            \item sistema mostra l’area riservata del paziente
            \item paziente clicca il bottone "Compila questionario"
            \item sistema mostra l'area riservata alla compilazione del questionario
            \item paziente inserisce i dati della terapia nei campi appositi
        \end{itemize}

        \medskip
        \textbf{Postcondizioni:}
    \end{tcolorbox}
    
    \medskip
    \begin{tcolorbox}[
        title = $\text{UC}_\text{p}$ - Riceve notifica,
        colback=white,
        colframe=black,
        fonttitle=\bfseries
    ]
        \textbf{Attori:} paziente
        
        \textbf{Precondizioni:} paziente ha effettuto correttamente l'auteniticazione al sistema
        
        \textbf{Flusso principale:}
        \begin{itemize}[label=$\triangleright$]
            \item paziente accede al sistema
            \item sistema mostra l’area riservata del paziente
            \item paziente riceve una notifica se ha una o più terapie in corso (da iniziare)
        \end{itemize}

        \medskip
        \textbf{Postcondizioni:}
    \end{tcolorbox}
    
    \medskip
    \begin{tcolorbox}[
        title = $\text{UC}_\text{p}$ - Invia mail,
        colback=white,
        colframe=black,
        fonttitle=\bfseries
    ]
        \textbf{Attori:} paziente
        
        \textbf{Precondizioni:} paziente ha effettuto correttamente l'auteniticazione al sistema
        
        \textbf{Flusso principale:}
        \begin{itemize}[label=$\triangleright$]
            \item paziente accede al sistema
            \item sistema mostra l’area riservata del paziente
            \item sistema mostra l'area riservata alla mail
            \item sistema consente al diabetologo di:
            \begin{itemize}[label=$\diamond$]
                \item inviare o rispondere a una mail
                \item inviare una mail al paziente che ha compilato in maniera errata il questionario
            \end{itemize}
        \end{itemize}

        \medskip
        \textbf{Postcondizioni:}
    \end{tcolorbox}

    \section{Activity diagram}

    \begin{figure}[H]
        \centering
        \includegraphics[width=0.5\linewidth]{../documentation/images/diagramma-attività-login.png}
        \caption{Autenticazione}
    \end{figure}

    \begin{figure}[H]
        \centering
        \includegraphics[width=1\linewidth]{../documentation/images/activity-diagram-diabetologo.jpeg}
        \caption{Diabetologo}
    \end{figure}

    \begin{figure}[H]
        \centering
        \includegraphics[width=1\linewidth]{../documentation/images/activity-diagram-paziente.jpeg}
        \caption{Paziente}
    \end{figure}

    % --- CAPITOLO 2 - SPIEGAZIONE DEL MODO DI PROCEDERE ---
    \chapter{Processo di sviluppo e pattern utilizzati}
    \section{Metodologia di sviluppo}

    In fase preliminare è stata condotta un'approfondita analisi dei requisiti, volta a identificare e formalizzare le funzionalità principali (\textit{core}) del progetto.

    Il processo di sviluppo del software ha poi seguito un approccio metodologico \textbf{agile e incrementale}. Nonostante la natura iterativa del processo, all'interno di ogni ciclo di sviluppo è stata mantenuta una rigorosa linearità operativa articolata nelle fasi di \textit{progettazione}, \textit{implementazione} e \textit{validazione}.

    Nello specifico, l'implementazione di ogni singolo caso d'uso (\textit{use-case}) è stata seguita da una fase di verifica puntuale (testing). In caso di esito negativo dei test, si è proceduto con attività di \textit{debugging} e \textit{refactoring}, iterando il processo fino al raggiungimento della stabilità e della conformità ai requisiti funzionali prefissati.

    A conclusione del ciclo di sviluppo, l'attività si è focalizzata sulla stesura della documentazione tecnica definitiva, arricchita dalla modellazione UML tramite diagrammi di attività e di sequenza, per rappresentare dettagliatamente il flusso logico del software.

    \section{Pattern Architetturali e Design Pattern}
    L'architettura del sistema è stata definita seguendo pattern consolidati nell'ingegneria del software, al fine di garantire modularità, manutenibilità e una chiara separazione delle responsabilità.

    \subsection{Model-View-Controller (MVC)}
    A livello macroscopico, il progetto adotta il pattern architetturale \textbf{MVC}. Questo ha permesso di disaccoppiare la logica di presentazione dalla logica di business e dai dati:
    \begin{itemize}
        \item \textbf{Model:} Rappresenta lo stato dell'applicazione.
        \item \textbf{View:} Si occupa della visualizzazione dei dati e dell'interazione con l'utente.
        \item \textbf{Controller:} Gestisce il flusso delle operazioni, ricevendo gli input dalla View e aggiornando il Model di conseguenza.
    \end{itemize}
    
    \subsection{Data Access Object (DAO)}
    Per la gestione della persistenza è stato implementato il pattern \textbf{DAO}. Questo pattern astrae e incapsula tutti gli accessi alla base di dati (CRUD), nascondendo i dettagli complessi delle query SQL al resto dell'applicazione. Grazie ai DAO, la logica di business non dipende direttamente dal database, facilitando eventuali migrazioni o modifiche alla struttura dati.

    \subsection{Singleton}
    Il pattern \textbf{Singleton} è stato utilizzato per garantire l'univocità di determinate istanze critiche all'interno dell'applicazione.
    \begin{itemize}
        \item Un esempio è la classe \texttt{Sessione.java}: il Singleton assicura che esista una sola istanza attiva relativa all'utente corrente, prevenendo conflitti di accesso ai dati di sessione.
        \item Un altro esempio è la classe \texttt{Navigator.java}: utilizzata per centralizzare la logica di navigazione, permettendo il cambio controllato delle viste (pagine) all'interno dell'interfaccia utente.
    \end{itemize}

    \subsection{Service Layer (Business Logic)}
    Per orchestrare le operazioni tra il Controller e il livello di persistenza (DAO), è stato introdotto un \textbf{Service Layer} (rappresentato dalla classe \texttt{AdminService}).
    Questa classe agisce come un punto di ingresso unificato per la logica di business: riceve le richieste dai controller, applica le regole di dominio necessarie e coordina le chiamate ai vari DAO. Sebbene tecnicamente definito come \textit{Service Layer}, questo componente applica il principio del \textbf{Facade Pattern}, semplificando l'interfaccia verso i sottosistemi di dati e nascondendo la complessità delle transazioni sottostanti.

    % --- CAPITOLO 3 - SPIEGAZIONE DEI TEST EFFETTUATI ---
    \chapter{Test}
    La fase di testing rappresenta una componente critica del ciclo di vita del software, essenziale per garantire la robustezza, l'affidabilità e la corretta esecuzione delle funzionalità implementate. In linea con la metodologia agile adottata, l'attività di verifica non è stata relegata alla sola fase finale, ma è stata integrata durante tutto il processo di sviluppo.

    La strategia principale adottata è stata quella dello \textbf{Unit Testing}. Questo approccio consiste nell'isolare e testare singole porzioni di codice (solitamente metodi o classi) per accertarsi che funzionino correttamente in autonomia, indipendentemente dal resto del sistema.

    Per l'implementazione automatizzata dei test è stato utilizzato il framework \textbf{JUnit}, lo standard industriale per l'ecosistema Java.

    \section{Test degli sviluppatori}
    \subsection{Test della Logica}
    In questa fase sono stati eseguiti test mirati a validare la logica applicativa, sottoponendo il sistema a diverse tipologie di input (validi, errati e limiti) per verificare la conformità della risposta rispetto al comportamento atteso.
    
    Tra i principali scenari di test effettuati si evidenziano:
    \begin{itemize}
        \item \textbf{Meccanismo di autenticazione:} Verifica della sicurezza in fase di login. È stato accertato che l'inserimento di credenziali errate o non presenti nel database comporti il blocco dell'accesso, reindirizzando l'utente alla schermata di errore appropriata.
        \item \textbf{Dashboard del Diabetologo:} Verifica del corretto recupero dei dati dal database e del popolamento dinamico delle liste pazienti e delle metriche nella Home Page dedicata al medico.
        \item \textbf{Sistema di notifica:} Validazione del modulo di comunicazione, con particolare attenzione al corretto invio e alla ricezione delle e-mail di avviso generate dal sistema.
        \item \textbf{Gestione Dati Pazienti (CRUD):} Test approfonditi sulle operazioni di inserimento, aggiornamento ed eliminazione dati. È stata verificata la robustezza del sistema a fronte di \textit{input} non validi (es. formati errati) o dati mancanti, assicurando che le eccezioni vengano gestite senza causare crash del software.
    \end{itemize}
    
    \subsection{Validazione dell'Interfaccia Grafica (GUI)}
    Oltre ai test automatici sulla logica, è stata eseguita un'estesa fase di \textbf{testing manuale} sull'interfaccia utente.
    L'obiettivo di questa fase è stato verificare la robustezza del layer di presentazione (View) e la corretta integrazione con il Controller.
    
    Nello specifico, è stata condotta una verifica esaustiva di tutti gli elementi interattivi (pulsanti, form) per accertarsi che:
    \begin{itemize}
        \item Ogni azione scatenata dall'utente (es. click su un pulsante di conferma) venisse gestita correttamente senza generare errori a runtime o crash dell'applicazione.
        \item La navigazione tra le diverse schermate avvenisse in modo fluido e coerente, senza "link rotti" o pagine irraggiungibili.
        \item Il sistema fornisse un feedback visivo adeguato ad ogni interazione (es. messaggi di conferma o alert di errore), garantendo una buona esperienza utente.
    \end{itemize}

    \section{Test utente generico}

\end{document}